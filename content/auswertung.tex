\section{Auswertung}
\label{sec:Auswertung}
\subsection{Absorbtion von \texorpdfstring{$\gamma$}{Gamma}-Strahlung}
Im Folgenden wird die Absorbtion von $\gamma$-Strahlung untersucht.
Für die Nullrate ergibt sich ein Wert von:
%
\begin{table}[H]
    \caption{Nullrate der $\gamma$-Strahlung.}
    \label{tab:gamma_null}
    \centering
    \begin{tabular}{S[table-format=3(0)e0] S[table-format=3(0)e0] S[table-format=1.2(0)e0] }
        \toprule
        {$t/\si{\second}$} & {Counts} & {Aktivität$/\si{\becquerel}$} \\
        \midrule
        900 &   951 & 1.06 \\
        \bottomrule
    \end{tabular}
\end{table}
Aufgrund der hohen Messzeit für die Nullrate ist der Fehler vernachlässibar gering.
\noindent
Die Messwerte für eine Abschirmung durch Blei sind in Tabelle \ref{tab:gamma_pb} zusammen mit der entsprechenden Aktivität aufgetragen.
\begin{table}[H]
    \caption{Messung der $\gamma$-Strahlung durch Blei.}
    \label{tab:gamma_pb}
    \centering
    \begin{tabular}{S[table-format=4(0)e0] S[table-format=3(0)e0] S[table-format=3.2(4)e0]  S[table-format=2]}
        \toprule
        {Counts} & {$t/\si{\second}$} & {Aktivität$/\si{\becquerel}$} & {$d/\si{\milli\meter}$} \\
        \midrule
             8096 & 60  & 133.88 \pm  1.50 &   1 \\
             7395 & 60  & 122.20 \pm  1.43 &   2 \\
             7272 & 70  & 102.83 \pm  1.22 &   3 \\
             8529 & 90  &  93.71 \pm  1.03 &   4 \\
             8105 & 100 & 79.99  \pm  0.90 &   5 \\
             6946 & 130 & 52.37  \pm  0.64 &  10 \\
             6278 & 150 & 40.80  \pm  0.53 &  12 \\
             4983 & 170 & 28.26  \pm  0.42 &  15 \\
             3946 & 220 & 16.88  \pm  0.29 &  20 \\
             5501 & 350 & 14.66  \pm  0.21 &  30 \\
             1557 & 500 &  2.06  \pm  0.09 &  40 \\
             1156 & 700 &  0.60  \pm  0.06 &  50 \\
        \bottomrule
    \end{tabular}
\end{table}
\noindent
Da die hier aufgenommenen Zählraten poissonverteilt sind ergibt sich eine Unsicherheit von $\sigma = \sqrt{N}$,
die Fortpflanzung geschieht nach Gauß und ist in Tabelle \ref{tab:gamma_pb} mit aufgenommen.
%
\begin{figure}[H]
  \centering
  \includegraphics[scale=0.9]{gamma_pb.pdf}
  \caption{Absorbtion der $\gamma$-Strahlung in Blei.}
  \label{fig:gamma_pb}
\end{figure}
\noindent
Es wird nach der Vorschrift
\begin{equation}
    y = e^{mx + n}
\end{equation}
mit den Fitparametern
\begin{align}
    \mu &= -m = \SI{0.099\pm0.001}{\per\milli\meter} \\
    N_0 &= n = \SI{4.94\pm0.01}{\becquerel}.
\end{align}
Im Vergleich dazu steht der nach Gleichung \eqref{eq:klein} berechnete Theoriewert:
\begin{align}
    \label{mu}
    \mu &= \sigma_\text{com}  \rho  N_A / Z \\
    \mu_\text{Theorie} &= \SI{0.069}{\per\milli\meter} \\
\end{align}
Hierbei ist $\rho$ die Dichte, $N_A$ die Avogadrokonstante \cite{na} und $Z$ die Ordnungszahl.
Es werden die Werte $\varepsilon = \num{1.295}$ \cite{v704}, $\rho = \SI{11.342}{\gram\per\centi\meter\cubed}$ \cite{blei_dichte} und $Z=\num{}$ verwendet.
% anzahl der teilchen pro volmeneinheit = dichte * teilchen/gramm = dichte * 6e23/Ordnungszahl
Somit ergibt sich eine Abweichung von $\SI{43.5\pm0.1}{\percent}$.
%
Für die Absorbtion in Zink ergeben sich folgende Messwerte:
\begin{table}
    \caption{Messung der $\gamma$-Strahlung durch Blei.}
    \label{tab:gamma_pb}
    \centering
    \begin{tabular}{S[table-format=5(0)e0] S[table-format=3(0)e0] S[table-format=3.2(4)e0]  S[table-format=2]}
        \toprule
        {Counts} & {$t/\si{\second}$} & {Aktivität$/\si{\becquerel}$} & {$d/\si{\milli\meter}$} \\
        \midrule
            11900 & 80 &  147.69 \pm   1.36 &      2        \\
            10454 & 80 &  129.62 \pm   1.28 &      4        \\
            10737 & 90 &  118.24 \pm   1.15 &      6        \\
            11149 & 100 & 110.43 \pm   1.06 &      8        \\
            11634 & 120 & 95.89  \pm   0.90 &     10        \\
            12375 & 140 & 87.34  \pm   0.80 &     12        \\
            12173 & 150 & 80.10  \pm   0.74 &     14        \\
            12027 & 160 & 74.11  \pm   0.69 &     16        \\
            11405 & 180 & 62.30  \pm   0.59 &     20        \\
        \bottomrule
    \end{tabular}
\end{table}
\noindent
\begin{figure}[H]
  \centering
  \includegraphics[scale=0.9]{gamma_zn.pdf}
  \caption{Absorbtion der $\gamma$-Strahlung in Zink.}
  \label{fig:gamma_zn}
\end{figure}
\noindent
Es wie zuvor linear
mit den Fitparametern
\begin{align}
    \mu &= -m = \SI{0.048\pm0.001}{\per\milli\meter} \\
    N_0 &= n = \SI{5.07\pm0.01}{\becquerel}
\end{align}
gefittet.
Im Vergleich dazu steht der nach Gleichung \eqref{eq:klein} berechnete Theoriewert:
\begin{align}
    \mu_\text{Theorie} = \SI{0.051}{\per\milli\meter} \\
\end{align}
Es werden die Werte $\varepsilon = \num{1.295}$ \cite{v704}, $\rho = \SI{7.14}{\gram\per\centi\meter\cubed}$ \cite{zink_dichte} und $Z=\num{}$ verwendet.
Somit ergibt sich eine Abweichung von \SI{5.9\pm2.0}{\percent}.
%
\subsection{Absorbtion von \texorpdfstring{$\beta$}{Beta}-Strahlung}
Aus der Messung der $\beta$-Stahlung entstehen die Messwerte in Tabelle \ref{tab:beta}
\begin{table}
    \caption{Messung der $\beta$-Strahlung durch Aluminium.}
    \label{tab:beta}
    \centering
    \begin{tabular}{S[table-format=5(0)e0] S[table-format=4(0)e0] S[table-format=3.2(4)e0]  S[table-format=3.0]}
        \toprule
        {Counts} & {$t/\si{\second}$} & {Aktivität$/\si{\becquerel}$} & {$d/\si{\micro\meter}$} \\
        \midrule
              636  & 1000 & 0.06 \pm  0.04  & 482 \\
              627  & 1000 & 0.05 \pm  0.04  & 444 \\
              646  & 1000 & 0.07 \pm  0.04  & 400 \\
              680  & 1000 & 0.10 \pm  0.04  & 338 \\
              627  & 900  & 0.12 \pm  0.04  & 302 \\
              534  & 700  & 0.19 \pm  0.04  & 253 \\
              864  & 500  & 1.15 \pm  0.06  & 200 \\
              1251 & 300  & 3.59 \pm  0.12  & 160 \\
              1336 & 200  & 6.10 \pm  0.18  & 153 \\
              1469 & 200  & 6.77 \pm  0.19  & 125 \\
              5573 & 200  & 27.29 \pm 0.37  & 100 \\
              46115 & 101 & 456.01 \pm 2.13 &   0 \\
        \bottomrule
    \end{tabular}
\end{table}
\noindent
\begin{figure}[H]
  \centering
  \includegraphics[scale=0.9]{beta.pdf}
  \caption{Absorbtion der $\beta$-Strahlung in Aluminium.}
  \label{fig:beta}
\end{figure}
\noindent
Es werden zwei Bereiche mit folgenden Wertepaaren linear gefittet:
\begin{align}
    m_1 &= \SI{-8.5\pm37.6}{\per\kg\meter\squared} \\ %e-10
    n_1 &= \SI{-1.8\pm4.4}{\becquerel}                  \\
    m_2 &= \SI{-112.3\pm0.4}{\per\kg\meter\squared} \\ %e-8
    n_2 &= \SI{6.125\pm0.005}{\becquerel}
\end{align}
Es folgt somit, dass
\begin{equation}
    R_\text{max} = \frac{n_2-n_1}{m_1-m_2}
\end{equation}
und
\begin{equation}
    \symup{\Delta}R_\text{max} = \sqrt{
        \left( \frac{1}{m_1-m_2} \right) ^2 (\symup{\Delta}n_1^2 + \symup{\Delta}n_1^2) +
        \left( \frac{n_2-n_1}{(m_1-m_2)^2}\right) ^2 \left(\symup{\Delta}m_1^2 + \symup{\Delta}m_2^2 \right)
        },
\end{equation}
wodurch ein Wert folgt von
\begin{equation}
    R_\text{max} = \SI{0.076\pm0.051}{\gram\per\centi\meter\squared}
\end{equation}
und somit durch Gleichung \eqref{emax}
\begin{equation}
    E_\text{max} = \SI{0.29\pm0.12}{\mega\electronvolt},
\end{equation}
wobei der Fehler mit
\begin{equation}
    \symup{\Delta}E_\text{max} = \frac{1.92 R_\text{max} + 0.2112}{\sqrt{R_\text{max}^2+0.22R_\text{max}}},
\end{equation}
folgt.
