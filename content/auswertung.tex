\section{Auswertung}
\label{sec:Auswertung}
\subsection{Absorbtion von \texorpdfstring{$\gamma$}{Gamma}-Strahlung}
Im Folgenden wird die Absorbtion von $\gamma$-Strahlung untersucht.
Für die Nullrate ergibt sich ein Wert von:
%
\begin{table}
    \caption{Nullrate der $\gamma$-Strahlung.}
    \label{tab:gamma_null}
    \centering
    \begin{tabular}{S[table-format=3(0)e0] S[table-format=3(2)e0] S[table-format=1.2(4)e0] }
        \toprule
        {$t/\si{\second}$} & {Counts} & {Aktivität$/\si{\becquerel}$} \\
        \midrule
        900 &   951\pm31 & 1.06\pm0.03 \\  
        \bottomrule
    \end{tabular}
\end{table}
\noindent
Da die hier aufgenommenen Zählraten Poissonverteilt sind ergibt sich eine Unsicherheit von $\sigma = \sqrt{N}$, 
die Fortpflanzung geschieht nach Gauß.
%
\begin{figure}[H]
  \centering
  \includegraphics[scale=0.5]{gamma_pb.pdf}
  \caption{Absorbtion der $\gamma$-Strahlung in Blei.}
  \label{fig:gamma_pb}
\end{figure}
\noindent
Es wurde nach der Vorschrift
\begin{equation}
    y = mx + n
\end{equation}
mit den Fitparametern
\begin{align}
    \mu &= m = \SI{}{\per\milli\meter} \\
    N_0 &= n = \SI{}{\becquerel}.
\end{align}
Im Vergleich dazu steht der nach Gleichung \eqref{eq:dubbi} berechnete Theoriewert:
\begin{align}
    \mu_\text{Theorie} = \SI{0.069}{\per\milli\meter} \\
\end{align}
Somit ergibt sich eine Abweichung von \SI{}{\percent}.
%
Für die Absorbtion in Zink ergeben sich folgende Messwerte:
%tabelle
\noindent
\begin{figure}[H]
  \centering
  \includegraphics[scale=0.5]{gamma_zn.pdf}
  \caption{Absorbtion der $\gamma$-Strahlung in Zink.}
  \label{fig:gamma_zn}
\end{figure}
\noindent
Es wie zuvor linear 
mit den Fitparametern
\begin{align}
    \mu &= m = \SI{}{\per\milli\meter} \\
    N_0 &= n = \SI{}{\becquerel}
\end{align}
gefittet.
Im Vergleich dazu steht der nach Gleichung \eqref{eq:dubbi} berechnete Theoriewert:
\begin{align}
    \mu_\text{Theorie} = \SI{0.051}{\per\milli\meter} \\
\end{align}
Somit ergibt sich eine Abweichung von \SI{}{\percent}.
%
\subsection{Absorbtion von \texorpdfstring{$\beta$}{Beta}-Strahlung}
Aus der Messung der $\beta$-Stahlung entstehen die Messwerte in Tabelle \ref{tab:beta}
%tabelle
\noindent
\begin{figure}[H]
  \centering
  \includegraphics[scale=0.5]{beta.pdf}
  \caption{Absorbtion der $\beta$-Strahlung in Aluminium.}
  \label{fig:beta}
\end{figure}
\noindent
