\section{Diskussion}
\label{sec:Diskussion}
Aus den Werten in Tabelle lässt sich entnehmen, dass der Absorptionskoeffizient für die Messung mit Blei um ungefähr $40\%$ abweicht und für Zink um ungefähr $6\%$.
Die Messergebnisse der $\beta$-Strahlen-Absorption divergieren von der Theoriekurve \ref{fig:abs} und es lässt sich für hohe Schichtdicken kein eindeutiges Plateau erkennen.
Die größere Ungenauigkeit für die Messung mit den Blei-Absorberplatten lässt sich zumindest teilweise auf den Zustand dieser zurückführen.
Diese wiesen keine einheitliche Oberfläche auf.
Hauptsächlich sollte der Fehler durch die hohe Ordnungszahl des Bleis zurückgeführt werden können, 
durch welchen eine Paarbildung wahrscheinlicher wird, jedoch wird diese hier nicht berücksichtigt.
Der relativ genaue Wert von Zink lässt sich lässt sich durch den besseren Zustand der Absorberplatten und die geringere Ordnungszahl zurückführen.
Der ermittelte Wert für die maximale Energie der $\beta$-Strahlung hat eine vergleichsweise große Unsicherheit.
dies könnte durch den geringeren Unterschied zu der Nullrate, besonders bei hohen Schichtdicken, erklärt werden, was auch eine Erklärung des fehlenden Plateaus des $\beta$-Absorptionsverlauf erklären könnte.
