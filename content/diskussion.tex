\section{Diskussion}
\label{sec:Diskussion}
Aus den Werten in Tabelle lässt sich entnehmen, dass der Absorptionskoeffizient für die Messung mit Blei um ungefähr $40\%$ abweicht und für Zink um ungefähr $8\%$.
Die Messergebnisse der $\beta$-Strahlen-Absorption divergieren von der Theoriekurve \ref{fig:abs} und es lässt sich für hohe Schichtdicken kein eindeutiges Plateau erkennen.
Die größere Ungenauigkeit für die Messung mit den Blei-Absorberplatten lässt sich zumindest teilweise auf den Zustand dieser zurückführen.
Diese wiesen keine einheitlich Oberfläche auf.
Der relativ genaue Wert von Zink lässt sich lässt sich durch denn besseren Zustand dieser Absorberplatten zurückführen.
Auffallend ist, dass für größere Schichtdicken der Fehler der einzelnen Messungen ansteigt.
Dies könnte durch den geringeren Unterschied zu der Nullrate erklärt werden, was auch eine Erklärung des fehlenden Plateaus des $\beta$-Absorptionsverlauf sein könnte.
