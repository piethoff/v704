\section{Zielsetzung}
Ziel dieses Versuchs ist es den Absorptionskoeffizienten von verschiedener Materialien zu bestimmen.
Dafür wird die Wechselwirkung von $\gamma$- und $\beta$-Strahlung mit diesen Materialien untersucht.
\section{Theorie}
\label{sec:Theorie}
\subsection{Absorption}
Trifft ein Teilchenstrahl auf Materie, so finden Wechselwirkungen mit dieser statt.
Der Wirkungsquerschnitt $\sigma$ stellt ein Maß für die Häufigkeit der Wechselwirkungen dar.
Die Wahrscheinlichkeit, dass ein Teilchen eine Reaktion auslöst beträgt für einen Absorber mit Querschnitt F und Dicke D:
\begin{equation}
  W= \frac{nFD\sigma}{F}= nD \sigma    .
\end{equation}
Wenn $N_0$ Teilchen auf den Absorber treffen, finden
\begin{equation}
  N = N_0 nD \sigma
  \label{eq:gl1}
\end{equation}
Wechelwirkungen pro Zeit statt.
Bei einem realen Absorber muss, aufgrund der Vielzahl der Atome pro Volumeneinheit, $\sigma$ über eine infinitesimal dünne Schicht $dx$ aufsummiert werden.
Es kann nicht davon ausgegangen werden, dass die $\sigma$ sich in Strahlrichtung nicht überdecken.
Es wird angenommen, dass diese dünne Schicht $dx$ an einer Stelle $x$ innerhalb des Absorbers, mit der Dicke D, befindet.
Dann finden gemäß Gleichung\eqref{eq:gl1}
\begin{equation}
  dN = -N(x) n \sigma dx
\end{equation}
Reaktionen statt.
Die Zahl der Teilchen, welche nach Durchgang durch den Absorber übrigbleiben, lässt sich durch Integration über alle Schichten dx finden.
Damit ergibt sich die Gleichung:
\begin{equation}
  N(D) = N_0 \exp(-n\sigma D)  .
\end{equation}

\subsection{Gamma-Strahlung}

\subsection{Beta-Strahlung}
